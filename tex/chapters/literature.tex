%!TEX root = ../main.tex

\chapter{Literature Review}\label{cha:literature_review}

\section{Background}

- Explain Prisoner's Dilemma

- Axelrod's original tournament

- Work to reproduce Axelrod's tournament ~\cite{Knight2016}

\begin{tabular}{c c c c}\label{tab:tournament_refs}
Year & Reference & Number of Strategies & Type\\
\hline
1979 & & 13 & Standard\\
1979 & & 64 & Standard\\
1984 & & 64 & Evolutionary\\
1991 & & 13 & Noisy\\
2005 & & 223 & Varied\\
2012 & & 13 & Standard\\
\hline
\caption{An overview of published tournaments}
\end{tabular}

is often used to model systems in biology ~\cite{Sigmund1999}, sociology ~\cite{Franken2005},
psychology ~\cite{Ishibuchi2005}, and economics ~\cite{Chong2005}.


\section{Fingerprinting}\label{sec:fingerprinting}

\begin{definition}\label{def:joss-ann}
If $A$ is a strategy for playing the iterated prisoner's dilemma, then the \textbf{Joss-Anne of A}, $JA(A, x, y)$ is a strategy which has a probablility $x$ of choosing move $C$, a probability $y$ of chosing move $D$, and otherwise uses the response appropriate to strategy $A$.
\end{definition}

The notation $JA$ comes from the initials of the names Joss and Anne.
Joss was a strategy submitted to one of Axelrod’s original tournaments and it would occasionally defect without provocation in the hopes of a slight improvement in score.
Anne is the first name of A. Stanley who suggested the addition of random cooperation (refs from ashlock paper) instead of random defection.
When $x + y = 1$, the original strategy is not used, and the resulting behavior is a random strategy with probabilities $(x, y)$.
In more general terms, a $JA$ strategy is an alteration of a strategy $A$ that causes the strategy to be played with random noise inserted into the responses.

\begin{definition}\label{def:fingerprint}
A \textbf{Fingerprint} $F_A(S, x, y)$ with $0 \leq x, y \leq 1$, $x+y \leq 1$ for strategy $S$ and probe $A$, is the function that returns the expected score of strategy $S$ against $JA(A, x, y)$ for each possible $(x, y)$.
\end{definition}

\begin{definition}\label{def:double-fingerprint}
The \textbf{Double Fingerprint} $F_{AB}(S, x, y)$ with $0 \leq x, y \leq 1$ returns the expected score of strategy $S$ against $JA(A, x, y)$ if $x+y \leq 1$, and $JA(B, 1-y, 1-x)$ if $x+y \geq 1$.
\end{definition}

\begin{definition}\label{def:dual}
Strategy $A'$ is said to be the \textbf{Dual} of strategy $A$ if $A$ and $A'$ can be written as finite-state machines that are identical except that their responses are reversed.
\end{definition}

\begin{theorem}\label{fingerprint-unit-square}
If $A$ and $A'$ are dual strategies, then $F_{AA'}(S, x, y)$ is identical to the function $F_A(S, x, y)$ extened over the unit square.
\end{theorem}


\section{Example Fingerprint Construction}

There are several steps to constructing the Fingerprint of a strategy a basic familiarity of Markov Chains is required.
An outline of the steps is as follows:

\begin{itemize}
    \item Build the markov chain for IPD between the strategy and probe strategy.
    \item Construct the corresponding transition matrix.
    \item Find the steady state distribution.
    \item Calculate the overall expected score by taking the dot product of the steady state distribution with the payoff vector.
    \item Plot the resulting function. 
\end{itemize}

We will now apply this process in order to obtain a fingerprint for the strategy Win-Stay-Lose-Shift (sometimes referred to as Pavlov) when probed by Tit-For-Tat.

\textbf{Step 1} - Build the markov chain.

\textbf{Step 2} - Construct the transition matrix.

\begin{equation}\label{eq:transition_matrix}
% 
T = \bordermatrix{~      & (C, C) & (C, D) & (D, C) & (D, D) \cr
                  (C, C) & 1 - y  & 0      & 0      & x      \cr
                  (C, D) & y      & 0      & 0      & 1 - x  \cr
                  (D, C) & 0      & 1 - y  & x      & 0      \cr
                  (D, D) & 0      & y      & 1 - x  & 0}
% 
\end{equation}

\textbf{Step 3} - Find the steady state distribution.

\begin{equation}\label{eq:steady_state}
%
\pi =
\begin{bmatrix}
\cfrac{x (1 - x)}{2y(1-x) + x(1-x) + y(1-y)}, \\
\cfrac{y (1 - x)}{2y(1-x) + x(1-x) + y(1-y)}, \\
\cfrac{y (1 - y)}{2y(1-x) + x(1-x) + y(1-y)}, \\
\cfrac{y (1 - x)}{2y(1-x) + x(1-x) + y(1-y)}
\end{bmatrix}
\end{equation}
\textbf{Step 4} - Calculate the expected score.

\begin{equation}
F = \cfrac{3x(1-x) + y(1-x) + 5y(1-y)}{2y(1-x) + x(1-x) + y(1-y)}
\end{equation}

\textbf{Step 5} - Plot the resulting function.