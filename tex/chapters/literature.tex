%!TEX root = ../main.tex

\chapter{Literature Review}\label{cha:literature_review}

\section{Background}

- Explain Prisoner's Dilemma

- Axelrod's original tournament

- Work to reproduce Axelrod's tournament ~\cite{Knight2016}

\begin{table}[htbp]
\centering
\begin{tabular}{c c c c}
Year & Reference & Number of Strategies & Type\\
\hline
1979 & & 13 & Standard\\
1979 & & 64 & Standard\\
1984 & & 64 & Evolutionary\\
1991 & & 13 & Noisy\\
2005 & & 223 & Varied\\
2012 & & 13 & Standard\\
\hline
\end{tabular}
\label{tab:tournament_refs}
\caption{An overview of published tournaments}
\end{table}

is often used to model systems in biology ~\cite{Sigmund1999}, sociology ~\cite{Franken2005},
psychology ~\cite{Ishibuchi2005}, and economics ~\cite{Chong2005}.


\section{Fingerprinting}\label{sec:fingerprinting}

\begin{definition}\label{def:joss-ann}
If $A$ is a strategy for playing the iterated prisoner's dilemma, then the \textbf{Joss-Anne of A}, $\JA(A, x, y)$ is a transformation of that strategy.
Instead of the original behaviour, it makes move $C$ with probablility $x$, move $D$ with probability $y$, and otherwise uses the response appropriate to strategy $A$ (if $x+y < 1$).
\end{definition}

The notation $\JA$ comes from the initials of the names Joss and Anne.
Joss was a strategy submitted to one of Axelrod’s original tournaments and it would occasionally defect without provocation in the hopes of a slight improvement in score.
Anne is the first name of A. Stanley who suggested the addition of random cooperation (refs from ashlock paper) instead of random defection ~\cite{Ashlock2008}.
When $x + y = 1$, the original strategy is not used, and the resulting behavior is a random strategy with probabilities $(x, y)$.
In more general terms, a $\JA$ strategy is an alteration of a strategy $A$ that causes the strategy to be played with random noise inserted into the responses.

\begin{definition}\label{def:fingerprint}
A \textbf{Fingerprint} $F_A(S, x, y)$ with $0 \leq x, y \leq 1$, $x+y \leq 1$ for strategy $S$ and probe $A$, is the function that returns the expected score of strategy $S$ against $\JA(A, x, y)$ for each possible $(x, y)$.
\end{definition}



\begin{definition}\label{def:double-fingerprint}
The \textbf{Double Fingerprint} $F_{AB}(S, x, y)$ with $0 \leq x, y \leq 1$ returns the expected score of strategy $S$ against $\JA(A, x, y)$ if $x+y \leq 1$, and $JA(B, 1-y, 1-x)$ if $x+y \geq 1$.
\end{definition}

\begin{definition}\label{def:dual}
Strategy $A'$ is said to be the \textbf{Dual} of strategy $A$ if $A$ and $A'$ can be written as finite-state machines that are identical except that their responses are reversed.
\end{definition}

An alternative wording is that, given a history for an opponent, the responses of the original strategy and the dual would be opposite.
It's important to note that this is different to taking a strategy and flipping it's responses.
The dual relies on knowledge of the underlying state of the original strategy, whereas the flip does not
This is shown in Table \ref{tab:strat-dual-flip}.

\begin{table}[htbp]
\centering
\begin{tabular}{c c  c | c}
Pavlov & Dual & Flip & Opponent \\
\hline
C & D & D & C\\
C & D & C & D\\
D & C & C & D\\
C & D & C & C\\
C & D & D & C\\
C & D & C & D\\
D & C & C & C\\
D & C & D & D\\
C & D & D &
\end{tabular}
\caption{The different responses of Pavlov, Pavlov's Dual and Flipped Pavlov}
\label{tab:strat-dual-flip}
\end{table}

The subtle difference between Dual and Flip can be highlighted further by inspecting each row individualy.

\underline{Row 1} - Pavlov always plays $C$ on the first go.
Flip will change this to $D$.
Dual knows that Pavlov always plays $C$ and so swaps to $D$.

\underline{Row 2} - In the previous round for Pavlov the strategies played $(C, C)$, and so Pavlov plays $C$ again.
For Flip, the preceding interaction was $(D, C)$, in this instance Pavlov would play $D$ again, so this gets flipped to $C$.
The previous turn for Dual was $(D, C)$ so it infers that Pavlov had $(C, C)$.
It knows that Pavlov would play $C$ and so plays $D$.

\underline{Row 3} - In the previous round for Pavlov the strategies played $(C, D)$, and so Pavlov would change to play $D$.
For Flip, the preceding interaction was $(C, D)$, in this instance Pavlov would change to $D$, so this gets flipped to play $C$ again.
The previous turn for Dual was $(D, D)$ so it infers that Pavlov had $(C, D)$.
It knows that Pavlov would play $D$ in this instance and so plays $C$.



\begin{theorem}\label{thm:fingerprint-unit-square}
If $A$ and $A'$ are dual strategies, then $F_{AA'}(S, x, y)$ is identical to the function $F_A(S, x, y)$ extended over the unit square.
\end{theorem}


\section{Example Fingerprint Construction}



\section{Finite State Machines}\label{sec:fsm}



\begin{theorem}\label{thm:fsm}
Given a determenistic strategy $\alpha$ and 2 histories $h_1, h_2$, then for all games of length $n \in 1,2,3,...$ there exists a FSM such that $\alpha(h_1, h_2)$ can be obtained from the FSM.
\end{theorem}

\begin{proof}\label{prf:fsm}
Let $\sigma = \{C, D\}$ and
\[
S = \bigcup_{i=0}^{n+1} \{C, D\}^i \times \{C, D\}^i
\delta((h_1, h_2), a) =()
\]

\end{proof}






