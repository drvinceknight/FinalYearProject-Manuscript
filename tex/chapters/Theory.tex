%!TEX root = ../main.tex

\chapter{Theory}\label{cha:theory}

\section{Analytical Fingerprints}
There are several steps to constructing the Fingerprint of a strategy and basic knowledge of Markov Chains is required.
An outline of the steps is as follows:

\begin{enumerate}
    \item Build the markov chain for IPD between the strategy and probe strategy.
    \item Construct the corresponding transition matrix.
    \item Find the steady state distribution.
    \item Calculate the overall expected score by taking the dot product of the steady state distribution with the payoff vector given in .
    %TODO
    \item Plot the resulting function.
\end{enumerate}

We will now apply this process in order to obtain a fingerprint for the strategy Win-Stay-Lose-Shift (sometimes referred to as Pavlov) when probed by Tit-For-Tat.

\textbf{Step 1} - Build the markov chain.

\textbf{Step 2} - Construct the transition matrix.

\begin{equation}\label{eq:transition_matrix}
%
T = \bordermatrix{~      & (C, C) & (C, D) & (D, C) & (D, D) \cr
                  (C, C) & 1 - y  & 0      & 0      & x      \cr
                  (C, D) & y      & 0      & 0      & 1 - x  \cr
                  (D, C) & 0      & 1 - y  & x      & 0      \cr
                  (D, D) & 0      & y      & 1 - x  & 0}
%
\end{equation}

\textbf{Step 3} - Find the steady state distribution.

\begin{equation}\label{eq:steady_state}
%
\pi =
\begin{bmatrix}
\cfrac{x (1 - x)}{2y(1-x) + x(1-x) + y(1-y)}, \\
\cfrac{y (1 - x)}{2y(1-x) + x(1-x) + y(1-y)}, \\
\cfrac{y (1 - y)}{2y(1-x) + x(1-x) + y(1-y)}, \\
\cfrac{y (1 - x)}{2y(1-x) + x(1-x) + y(1-y)}
\end{bmatrix}
\end{equation}
\textbf{Step 4} - Calculate the expected score.

\begin{equation}
F = \pi \cdot
\begin{bmatrix}
3 \\
0 \\
5 \\
1
\end{bmatrix}
=
\cfrac{3x(1-x) + y(1-x) + 5y(1-y)}{2y(1-x) + x(1-x) + y(1-y)}
\end{equation}

\textbf{Step 5} - Plot the resulting function.

\section{Finite State Machines}

Figure \ref{fig:Tit4TatFSM} and figure \ref{fig:PavlovFSM} show the Finite State Machine (FSM) representations for Tit-For-Tat and Pavlov respectively.
Nodes represent the previous action taken by the strategy and the opponent, ie node $(D, C)$ implies that is the preceding turn, the strategy chose to Defect and the opponent chose to co-operate.
Arcs represent the choice made by the opponent at the current turn, and lead us to the state of the next turn.

It should be noted that these figures are not necessarily the simplest representation of their corresponding strategy.
For example, Tit-For-Tat requires no knowledge of its own previous moves, but they have been included for completeness.

\begin{figure}[!hbtp]
    \begin{center}
        \includestandalone{../img/Tit4TatFSM}
        \caption{FSM for TitForTat}\label{fig:Tit4TatFSM}
        \includestandalone{../img/PavlovFSM}
        \caption{FSM for Pavlov (Win-Stay Lose-Shift)}\label{fig:PavlovFSM}
    \end{center}
\end{figure}

\begin{figure}[!hbtp]
    \begin{center}
        \includestandalone{../img/MajorityFSM}
        \caption{FSM for Majority in a game with 4 Turns}\label{fig:MajorityFSM}
    \end{center}
\end{figure}

In figure \ref{fig:MajorityFSM} we have a more complex FSM for the strategy Majority which plays in the following way:

\begin{itemize}
  \item If the opponent has cooperated the majority of the time, Majority will cooperate
  \item If the opponent has defected the majority of the time, Majority will defect
  \item Note - the strategy shown is technically Soft Majority, if the opponents cooperations and defections are equal it will cooperate. Hard Majority would defect in this situation.
\end{itemize}

This implies that the strategy Majority requires knowledge of all previous states, and therefore could not be represented as an FSM.
However in Theorem \ref{thm:fsm} it is shown that if the number of turns in a game is known, any strategy can be represented as an FSM. A formal defintion of a Finite State Machine is given by Definition \ref{def:fsm} but first we will outline some motivating key characteristics of a system that can be modeled with a FSM:

\begin{itemize}
 \item The system must be describable by a finite set of states.
 \item The system must have a finite set of inputs that can trigger transitions between states.
 \item The behavior of the system at a given point in time depends upon the current state and the input that occurs at that time.
 \item For each state the system may be in, behavior is defined for each possible input.
 \item The system has a particular initial state.
\end{itemize}

\begin{definition}\label{def:fsm}
A \textbf{Deterministic Finite State Machine} $M$ is a tuple $(S, \sigma, \delta, s_0, F)$ where
\begin{itemize}
 \item $\sigma$ is the set of symbols representing the input of $M$.
 \item $S$ is the set of states of $M$.
 \item $s_0 \in S$ is the starting state.
 \item $F \subseteq S$ is the set of final states of $M$.
 \item $\delta: S \times \sigma \rightarrow S$ is the transition function.
\end{itemize}
\end{definition}





\begin{theorem}\label{thm:fsm}
Given a determenistic strategy $\alpha$ and 2 histories $h_1, h_2$, then for all games of length $n \in 1,2,3,...$ there exists a FSM such that $\alpha(h_1, h_2)$ can be obtained from the FSM.
\end{theorem}

\begin{proof}\label{prf:fsm}
Let $\sigma = \{C, D\}$ and
\[
S = \bigcup_{i=0}^{n+1} \{C, D\}^i \times \{C, D\}^i
\delta((h_1, h_2), a) =()
\]

\end{proof}



\lipsum
