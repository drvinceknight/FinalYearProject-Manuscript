%!TEX root = ../main.tex

\chapter{Introduction}\label{cha:introduction}

The Prisoner's Dilemma (PD) is a classical model in Game Theory.
It is a simple two player game where the agents must make a choice of two options without communicating.
This is often presented as two suspects who have been arrested and are being interrogated separately.
They have the option of whether to cooperate with each other or defect and each player receives an individual payoff that depends on the actions that have been taken.

Consider the situation that both prisoners return to their cells each night, to be presented with the same choice the next day.
Furthermore, allow this process to continue repeatedly.
This is called the Iterated Prisoner's Dilemma (IPD) and has been an object of interest ever since the 1950's but became more popular after Robert Axelrod's work in the 1980's \cite{Axelrod2016}.
Every player in IPD will have a strategy.
A strategy is a predetermined program of play that tells the player what actions to take in response to every possible strategy other players might use.

Several computer based IPD tournaments have been held over the years, but the first were held by Axelrod in 1980 (use our paper for refs). %TODO
Others worth noting are the two anniversary tournaments  held in 2004 and the Stewart and Plotkin 2012 tournament.
The original source code of these is only available for Axelrod's second tournament (in FORTRAN) and it is not well documented, tested or easily re-usable.

Recently, a group of scientists have been working to improve this situation by producing an open source version of Axelrod's original tournament \cite{Knight2016}.
Referred to as the Axelrod-Python library, it aims to provide a resource for the design of new strategies and interactions between them, as well as conducting tournaments and ecological simulations for populations of strategies.
The version of the library used in this work will be \cite{axelrodproject}, the library currently has 139 strategies implemented with the capability to run evolutionary tournaments and tournaments on different topologies.

\section{Prisoner's Dilemma}
The first formal definition of PD was presented by Albert W. Tucker during a seminar at Stanford University \cite{Gass2005}.
However, the idea was first formulated by the Merrils \cite{Flood1958} in 1950 whilst working for the RAND cooperation.

A description of the PD - Two players simultaneously decide whether to Cooperate $(C)$ or Defect $(D)$, without exchanging information.
They receive payoffs as follows:

\begin{itemize}
\item They both choose $C$ (mutual cooperation) and receive a payoff $R$ (Reward)
\item They both choose $D$ (mutual defection) and receive a payoff $P$ (Punish)
\item One player chooses $C$ and the other chooses $D$. The cooperator receives a payoff $S$ (Sucker) and the defector receives a payoff $T$ (Temptation).
\end{itemize}

Figure \ref{eq:payoff_matrix_symb} shows the payoff matrix.

\begin{equation}\label{eq:payoff_matrix_symb}
%
P = \bordermatrix{~ & C & D \cr
                  C & (R, R) & (T, S) \cr
                  D & (S, T) & (P, P) \cr}
%
\end{equation}

There are also two assumptions that need to be stated.
Firstly, both players are rational.
Secondly, there is no communication between them.
It is then easy to see that regardless of the choice of one player, the other will always obtain a higher payoff by defecting instead of cooperating.
Therefore we have a pure Nash Equilibrium where both players defect, despite the fact that both players would do better if they were to cooperate with each other.
Additionally, to ensure this behaviour occurs, the payoffs must satisfy the following inequalities:

\begin{equation}\label{eq:intuitive}
S < D < C < T
\end{equation}

and

\begin{equation}\label{eq:ensure_coop}
(S + T) < 2 C
\end{equation}

Equation \ref{eq:intuitive} merely fixes the payoffs in their intuitive order.
Equation \ref{eq:ensure_coop} ensures that alternating between cooperating and defecting (players take it in turns to stab each other in the back) performs no better than mutual cooperation.
These inequalities allow for many different payoff matrices to be formulated, but values of $(R, S, T, P) = (3, 0, 5, 1)$ are commonly used in literature (find references). %TODO
We can now formulate the new payoff matrix, as shown in Figure \ref{eq:payoff_matrix}
A more detailed explanation of the Prisoner's Dilemma is given in \cite{Gotts2003}.

\begin{equation}\label{eq:payoff_matrix}
%
P = \bordermatrix{~ & C & D \cr
                  C & (3, 3) & (5, 0) \cr
                  D & (0, 5) & (1, 1) \cr}
%
\end{equation}

\section{Problem Description}

As previously mentioned, the Axelrod-Python library contains 139 strategies, significantly more than the 13 and 64 strategies submitted to Axelrod's first and second tournaments.
This now raises the issue of duplication, and subsequently, how to differentiate between strategies.
Currently the only known approach to this problem in literature is Fingerprinting which produces a visual representation of a strategy, an example is shown in Figure. %TODO

The issue with this method is that it relies on knowledge of the underlying Markov chain of the strategy which cannot be easily constructed from the source code.
In this paper an algorithm is presented that allows the construction of a fingerprint for any strategy.
Example code is also given that shows how this has been implemented within the Axelrod library.

%TODO - a new way of telling strategies apart...signiture?

\section{Structure}

Throughout this paper examples and demonstrations of code will be given.
Some of this code will be directly copied from the Axelrod-Python source code, at other times it will have been written specifically for this paper.
In order to help the reader differentiate between each scenario, two different colour schemes have been used.
The first, for Axelrod-Python source code is shown in listing \ref{lst:examplesourcecode} and the second, for any bespoke code, is shown in listing \ref{lst:exampleexamplecode}.

\begin{listing}[hbtp!]
\begin{SourceCode}
def function_name(parameters):
    """
    multiline comment
    """
    function_code = 4
    return some_output
\end{SourceCode}
\caption{An example of how Axelrod-Python source code will be displayed}
\label{lst:examplesourcecode}
\end{listing}

\begin{listing}[hbtp!]
\begin{ExampleCode}
def function_name(parameters):
    """
    multiline comment
    """
    function_code = 4
    return some_output
\end{ExampleCode}
\caption{An example of how demonstrative code will be displayed}
\label{lst:exampleexamplecode}
\end{listing}{}

This report is organized into several chapters. Continuing from this introduction:

\begin{itemize}
    \item Chapter \ref{cha:literature} An overview of previous literature is given
    \item Chapter \ref{cha:theory} an example analytical fingerprint is constructed
    \item Chapter \ref{cha:results} example code, the algorithm and comparison of fingerprints
    \item Chapter \ref{cha:conclusion}
\end{itemize}

